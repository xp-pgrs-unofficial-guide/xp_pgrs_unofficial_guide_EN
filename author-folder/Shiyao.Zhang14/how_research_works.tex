\section{How does Research Work}
\label{section.how-research-works}

\subsection{Funding}
Academic funding generally comes from the following three sources:
\begin{itemize}
    \item Government Agencies and Foundations: In China, there are grants available from the central government, provinces, and cities, aimed at both projects and individuals. Similarly, abroad, there are government-supported societies that provide grants. These projects usually have large amounts of funding and high impact. A successful application record in this system can significantly help in applying for similar projects in the future.
    \item Corporate Collaboration Projects: These are typically obtained by project leaders through negotiations with companies, and most of these projects focus on practical implementation. The amount of funding, completion rate, and impact can vary largely.
    \item Internal Research Support Projects (such as RDF): These are applied for by faculty members within the school, screened by the college, and reviewed by a specific committee. The number of projects is limited, the funding amounts are small, and the duration is short, but they are important for early-career researchers.
\end{itemize}

Academic funds do not go directly into the project leader’s personal bank account but into the school’s research fund management account. Expenditures must comply with specified spending categories and are subject to school regulations, commonly including procurement, conference-related expenses, labor costs, etc.

When academic funding exceeds a certain amount, the supervisor can obtain corresponding PostGraduate Research student Scholarship (PGRS) slots based on the amount of funding.


\subsection{Journals and Conferences}
Journals and conferences require at least one academic organization to host the event. Important and active researchers often take on roles such as editors and reviewers within these organizations, which helps attract submissions and participation. Correspondingly, taking on roles in academic organizations reflects a researcher’s reputation to some extent.

Publishers need payment to publish journals and host conferences. Generally, the more important, influential, and valuable a journal or conference is, the easier it is to find sponsors, the less authors need to pay for publication fees, and the higher the scale, richness, and quality of the event.

The impact of journals and conferences is largely influenced by the citation count of published articles. More trendy topics, significant work, well-known authors, and important journals are more likely to be read and cited. To ensure sustainability, organizers also prioritize these factors.

In many research fields, the quantities of researchers is small, so attending conferences often means meeting people who you have at least read their papers, and possibly even publication reviewers or viva examiners.


\subsection{Social Benefits}
One aspect of academic research is creating new knowledge at the edge of human cognition. From this perspective, academic research will ultimately benefit society.

Additionally, it is said that the humanities and social sciences also like to focus on social impact. We might invite someone to write about this in the future.

However, considering we live in a commercial society, social benefits often equate to the monetary value or potential of the results. This is also the environment we find ourselves in physically.


\subsection{Supervisors}
Different supervisors have different characteristics, but they can roughly be divided into application-oriented and frontier-oriented:
\begin{itemize}
    \item Application-Oriented supervisors: These advisors generally have more practical projects from enterprises or government, or application-oriented grants from the government.
    \item Frontier-Oriented supervisors: These advisors usually have more significant work published.
\end{itemize}

\vspace{\baselineskip}

These are not mutually exclusive; it’s just that time and energy are limited, so there will always be a certain inclination. Relatively speaking, there are more opportunities in application-oriented projects, while frontier-oriented projects require larger funding amounts. Due to the different inclinations of advisors, the resources they develop also differ, and the guidance you receive will vary.

Based on the above reasons, you might agree that we are likely to become like our supervisors.

For the humanities and social sciences, it is said that almost full-time supervisory is required. We might invite someone to write about this in the future.

\subsection{Teamwork}
\label{subsection.teamwork}
When conducting academic research, we may inevitably collaborate with other researchers for various reasons, including:
\begin{itemize}
    \item Research Funding: Well, since they provide the funding, it makes sense for them to give some input and be listed as authors\dots
    \item Academic Influence: Interestingly, experience suggests that articles involving prominent researchers are more likely to be accepted\dots
    \item Private Data: In many fields, undisclosed data is an unavoidable norm\dots
    \item Expertise: Many studies are interdisciplinary, and one might not understands all the details. Finding a researcher familiar with a particular aspect of the work can be very helpful, saving you a lot of effort and improving research efficiency. Of course, choosing the right person requires careful consideration\dots
    \item Experimental Work: Similar to expertise, suitable collaborators can save you a lot of effort. It is also common to hear about people enlisting undergraduates to help with experiments and co-authoring publications\dots
\end{itemize}

\vspace{\baselineskip}

Additionally, before you start collaborating with researchers outside your advisor’s team, be sure to read the school’s \href{https://ebridge.xjtlu.edu.cn/urd/sits.urd/run/SIW_FILE_LOAD.start_url?08F2CBCAE3174A7365Px9_5kBfG0_iGiWj8zb7ybwaO0YBYc8NiKlPG93xyQA9X2SClXOLLd7-_EgF50aijROwT-rdSGIUIbRRzhFu-76Ha0g2HymUr0S-Fgjm1DXP9RO1GhGzx5-akgsDSMBlNhR7vpib85F9vlqa67My7RKHFSiluZueFy52YCBtintt0wDTKmx4fCjkWnldNDaxo6ZVD2L572Us3V-FOv485wYZUNUn5NLzgR0pAaU7aiKnTVJY8Aa2su5F4u7o-rNPJPety3jwwJ4O1v1agpDZLZ1it1H5fWn3IgLNaWlUh84YpxNRXyTW1kIwrX0r4-dT1eoxdZUAFrJhaBwz6E0w}{Guidance on Authorship and Affiliation}, which might save you a lot of trouble.

\begin{flushright}
    October 1, 2024 by \Shiyao \\
\end{flushright}