\section{Remote Control: How to Access Campus Computers from Off-Campus}

When you are off-campus, you might often miss your campus computer. Whether it's the desktop provided by the school or other computers bought with your advisor's funds, they can all be remotely controlled.

\subsection{Preparation for School-Provided Computers}
(If you are not trying to access a school-provided computer, please skip to the next section)

School-provided computers are quite special, and by default, we cannot install software on them. There are two ways to handle this:
\begin{enumerate}
    \item Email or call IT and directly ask for administrator privileges. Just say you are a PhD student and need to install software on the school-provided computer. After that, you can install the control software.
    \item After the above method, the computer is still managed by the school and there are still some restrictions, but it is generally sufficient for regular use. For those who want complete control over the school computer, you can format the disk and reinstall the system, or partition the disk (keeping the original system) and install a dual system. There are many tutorials on reinstalling Windows, partitioning, and setting up dual systems on Bilibili and Zhihu.
\end{enumerate}

\subsection{Recommended Desktop Control Software}
The following software can be used on any system: Windows, Linux, or Mac.
\begin{itemize}
    \item VNC: Install VNC Server on the controlled end (school computer) \url{https://www.realvnc.com/en/connect/download/vnc/}, and VNC Viewer on the controlling end (your computer) \url{https://www.realvnc.com/en/connect/download/viewer/}. Register an account and log in to use, no need to remember connection codes. Recommended because VNC is a well-tested remote desktop solution that generally works as long as the network is stable.
    \item ToDesk: \url{https://www.todesk.com/} Install the same software on both computers. Recommended because it is a new software and generally smoother than VNC.
    \item Others: AnyDesk, TeamViewer, Sunflower, all can be tried. However, TeamViewer is not highly recommended because it might detect the network environment of the school and force you to purchase a commercial version, which is not an issue with other software.
\end{itemize}

\subsection{Remote SSH}
(If you don't use Linux, you can skip to the next section on pitfalls)

If the controlled end is a Linux system and you mainly use command-line operations, connecting via SSH directly will be much faster than desktop control.

The computer on campus has an internal IP starting with 10. How to SSH from outside? This trick is called "NAT traversal."

First, I recommend this video:

\href{https://www.bilibili.com/video/BV1Qq4y1w7F5}{[Hardcore] Public Network Access? NAT Traversal! Zero Experience Start!}\url{https://www.bilibili.com/video/BV1Qq4y1w7F5}

The solutions available for campus machines in the video are:
\begin{itemize}
    \item At 04:52 in the video, using IPV6 connection. However, note that in my experience, the school's IPV6 is unstable and might stop working after a few days. Even if you can set up DDNS, I personally do not recommend it.
    \item A more reliable solution is introduced at 08:09 in the video: NAT traversal. Free and easy-to-use solutions include: Zerotier (slightly slow as it is overseas), Peanut Shell (an old domestic brand, but registration requires an ID card), NOFRP (new, reliability might be low), or directly search "free NAT traversal" on Bilibili for many new solutions. If you are not familiar with these, there are many tutorials on Bilibili and Zhihu.
    \item (Advanced high-traffic version, but with a higher learning curve) Free solutions have limitations on traffic and bandwidth, which are sufficient for SSH commands. But if you want to use SCP to transfer files or forward VNC via SSH, consider renting a cloud server with your advisor's funds and manually setting up an FRP service (a bit complicated, but very smooth once set up, faster than the previous remote desktop solutions). Alternatively, set up a junk machine as a jump server in your dorm with a public IP and DDNS for unlimited traffic access. These solutions have a higher learning curve, but you can refer to online tutorials and experiment slowly.
\end{itemize}

\subsection{Pitfalls}
Here are some experiences I gained after encountering many pitfalls:
\begin{enumerate}
    \item Redundancy to improve reliability: Do not rely on a single remote control solution. If one fails, you can use another. Please install at least two, and if you are away for a long time, install more. I personally use: a second-hand thin client at home with FRP and DDNS + Peanut Shell as a backup plan + VNC as another backup plan + ToDesk, a four-layer backup solution for stable access to my campus Linux computer.
    \item After installation, restart once to see if the control software can start automatically.
    \item If the controlled end is a Windows computer, be sure to disable system updates. Although restarting is fine, a major Windows update might get stuck on a screen asking you to agree to a new user agreement, and unless you are there to click it, no software will start, and you will lose control. Please search "disable Windows updates" on Baidu. I personally recommend using a combination of group policy and host file modification.
    \item For those running programs, be sure not to exceed memory limits or crash the system memory. If the memory is fully used, the system will crash immediately, killing all control software, and it will not restart automatically. You must be there to force a restart. Please find ways to monitor and control memory usage in your programs.
    \item If possible, purchase remote KVM hardware like "Sunflower Control" to force a remote restart even if the system crashes.
    \item No matter how good the solution is, it cannot prevent network maintenance or power outages on campus. These can be defended against. For network outages, make sure to check the macauth option when logging in to the network, so it will work again after maintenance. For power outages, the motherboard needs to support the "boot on power" or "boot with power" function. Check the motherboard manual or search for "power" to see if it is available. Alternatively, purchase Sunflower Control. However, power and network outages are rare, and unless you are away for months, you generally do not need to worry about them.
    \item Maintain a good relationship with at least one on-campus classmate. Newcomers will encounter various unexpected situations and lose control, requiring office classmates to help check the machine. Behind 100 remote control solutions, you also need good classmates for physical control. In case the network cable is accidentally kicked out, you still need a classmate to help you fix it.
\end{enumerate}

\begin{flushright}
(09 November 2022 by \Wu) \\
Translated by GPT
\end{flushright}
