\subsection{Thesis Defense}

\subsubsection{Pre- and Post-Defense Process}

\begin{enumerate}
    \item At least two months before officially submitting your thesis on e-bridge, have your supervisor select a list of potential examiners for your upcoming defense. Submit this list to XJTLU's pgsupport and the University of Liverpool for review and approval.
    \item After officially submitting your thesis on e-bridge, the university will contact the respective examiners and then reach out to you to confirm the defense date and format (the defense typically occurs within 1-3 months after submission. The format can be either online or offline — confirm specifics with pgsupport).
    \item Generally, there will be a rehearsal before the defense to test the network and PPT presentation. You can also request time to practice your presentation in the designated conference room.
    \item Before the defense, the examiners will hold an internal meeting, usually on the same day (though sometimes earlier). They will then invite you into the meeting room to officially start the defense. Besides the two (or three under special regulations) examiners, there will be an observer (responsible for overseeing the overall process, arranging breaks, and addressing any procedural questions you might have). During the formal defense, the observer typically does not speak. If you have any questions, feel free to ask them.
    \item After your PPT presentation, there will be a Q\&A session. Depending on the circumstances, this can last from 1.5 to 3.5 hours (historically, there have been instances lasting up to 7 hours). Once the Q\&A concludes, you will be asked to leave the room temporarily while the examiners deliberate. After about 10-15 minutes, you'll be invited back in and informed of the results.
    \item The examiners' revision report will be sent to you within 10 working days. Based on their feedback, you'll need to make the necessary revisions within the specified timeframe and submit them to the primary examiner for confirmation.
    \item Once everything is confirmed, upload the final version of your thesis to complete the process.
\end{enumerate}

\begin{flushright}
    July 5, 2024 by Ziwen Xie \\
    GPT translation proofread by \Shiyao
\end{flushright}

\subsubsection{How to Prepare for a Viva}

Experience from the School of Science:

\textit{Try not to cram everything into a couple of days. Spreading out your preparation helps maintain a good state of mind daily, which is crucial for performing well during the actual defense.}

\textbf{1. Thoroughly read your thesis word by word as if you are a new reader, considering the following questions as you go:}
\begin{enumerate}
    \item Are any paragraphs unclear?
    \item Can you fully grasp and independently explain specific concepts?
    \item Do you understand the relationships and differences between chapters and the internal logic of the overall structure?
    \item Are you aware of the internal logic between sections within chapters and the sequence of experimental designs?
    \item What issues can each experiment address individually or collectively? What conclusions can be drawn from each major chapter?
    \item Do you fully understand the algorithms, models, and basic concepts cited from the literature? How are they connected to your core research objectives?
    \item Are there any previously unnoticed expression errors or chart inaccuracies that need correction?
    \item What is the purpose of your research? What motivated the choice of your methodology? How does it have advantages over other methods?
\end{enumerate}

\textbf{2. After organizing these points, analyze each chapter (including the abstract) in detail:}
\begin{enumerate}
    \item Can you summarize the background information in your own words?
    \item What does each subsection convey? Ensure consistency among subsections within a chapter.
    \item Highlight the key points or core logic of specific experiments or experimental designs.
    \item Understand the progression, parallelism, or other relationships and differences among experimental results (e.g., Experiment A demonstrates 'a', which serves as the foundation for Experiment B—this is progressive. If 'a' and 'b' are similar, then A and B are parallel).
    \item Justify the selection of experimental subjects and models (including why other related subjects or models weren't used).
    \item Based on your results, what future experiments could be pursued? (This may include why certain experiments weren't conducted, reasons for future experiments, potential outcomes of future experiments, foundational assumptions, and practical challenges—which explain why they weren't done in the current phase but were considered).
    \item Ensure the abstract includes key results and innovations.
\end{enumerate}

\textbf{3. Key Points for Preparing Your PPT and Viva:}
\begin{enumerate}
    \item Allocate 20 minutes into four sections: background introduction, experimental setup, experimental results, and conclusions—in approximately 5 minutes each. You can shorten the first three sections since the examiners have already read your thesis.
    \item Print your thesis and bring it with you to the defense. Take notes during the Q\&A session (this also gives you brief moments to compose yourself).
    \item Include essential figures and results in your PPT with concise descriptions.
    \item You don't need to discuss the challenges you faced or focus on future experiments or designs (examiners may ask about these, or you might mention them when answering questions). Depending on the situation, you can omit this from your PPT to save time.
    \item Ensure your PPT runs smoothly, and all text and images are clear. During practice sessions, simulate the actual presentation scenario as closely as possible.
    \item Remember to drink water during the defense—sip slowly and frequently. Eat some high-energy, easily digestible food beforehand. If you're not hungry, consider a chocolate bar or an energizing drink. Arrive early to familiarize yourself with the environment, which can help calm your nerves (and don't forget to visit the restroom beforehand).
    \item When answering questions, feel free to reference any part of your thesis. Some points in earlier chapters may connect with later ones or relate to future plans and final conclusions. If an examiner asks something you're unsure about or haven't considered, it's okay to admit it. Express your willingness to explore it further and ask for guidance or references in their feedback to assist with your revisions.
    \item Typically, each question and answer lasts about 3-5 minutes. With two examiners, handling around 20 questions each is substantial. Usually, the defense lasts between two to two and a half hours, with breaks in between (but once you're engaged, you might lose track of time).
\end{enumerate}

\textbf{4. Non-Thesis Related Questions:}
\begin{enumerate}
    \item How can artificial intelligence and ChatGPT inspire your research?
    \item Can your research be applied to other types of data, diseases, or fields? Why?
    \item What do you think are the shortcomings in your field?
    \item What is your biggest takeaway from your doctoral studies?
    \item If you had to do it all over again, would you still choose to pursue a Ph.D. and undertake this project?
\end{enumerate}
Tips: It seems the examiners are just curious and want a chat. Relax and answer accordingly. \textbf{Finally, according to data from a pgsupport staff member, there has never been a failed defense since our university was established. So, don't worry too much—just perform as you normally would. Good luck.}

\textbf{5. Ask your supervisors if they can help you anticipate potential questions. If time permits, see if they can arrange a mock viva for you (optional). It's also beneficial to consult recent graduates about their defense experiences, as certain aspects may change yearly.}

\begin{flushright}
    July 5, 2024 by Ziwen Xie \\
    GPT translation proofread by \Shiyao
\end{flushright}
