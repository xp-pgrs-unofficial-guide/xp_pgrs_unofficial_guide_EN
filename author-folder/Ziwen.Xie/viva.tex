\subsection{论文答辩}

\subsubsection{答辩前后流程}

\begin{enumerate}
    \item 在e-bridge上正式提交论文之前至少两个月,由指导老师选定好未来答辩的考察老师名单,交由XJTLU的pgsupport以及利物浦审核确定。
    \item 在e-bridge上正式提交论文后,学校会联系对应的考察老师,之后再联系博士生候选人确定答辩时间以及答辩方式(答辩时间一般在提交论文的1-3个月内,答辩方式分为线上线下两种,具体可以跟pgsupport沟通确认)。
    \item 一般来说,答辩之前会有一次预演,可以用来测试网络以及ppt展示。也可以另外申请时间去对应的会议室讲演ppt。
    \item 答辩前,考察老师们会先内部开一个通气会,一般来说是当天(但也有提前时间的),然后再邀请你进入会议室正式开始答辩。除了两位(或三位(特殊规则下))考察老师外,还有一位观察老师(负责整体流程推进和休息安排,并回答你的流程相关问题,在正式答辩阶段,基本上不会说话),如果有任何问题,都可以问他。
    \item ppt讲完是问答流程,根据具体情况,时间在一个半到3个半小时是比较常见的情况(历史上也出现过7个小时的)。问答流程结束后会请你离开会议室一段时间,老师们进行结果讨论。大约10-15分钟后,请你进去并现场告知结果。
    \item 考察老师的修改意见报告会在10个工作日之内发给你,根据具体的情况你需要在规定的时间内进行修改并交由主审考察老师确认。
    \item 确认无误之后把最终版本的论文上传,这个流程就算是结束了。
\end{enumerate}

\begin{flushright}
    (2024年07月05日 by Ziwen Xie)
\end{flushright}


\subsubsection{如何准备好一场答辩}

来自理学院的经验:

\textit{尽量不要堆在两三天内做这些,把内容稍微分散一点,利于每天保持好状态,维持好状态进入正式答辩相当重要。}

\textbf{一. 至少完整的逐字逐句的读完自己的thesis一遍,假设自己是个全新读者,边读边思考下列问题}
\begin{enumerate}
    \item 某些段落会不会不够清楚
    \item 某些概念是否自己已经完全掌握能独立解释
    \item 章节与章节之间的联系与区别自己是否明白,以及整体文章排布顺序的内在逻辑
    \item 章节内部的小段落,实验设计顺序之间的内在逻辑
    \item 每个或者某几个实验能一起或者分别说明什么问题,每个大章节能总结出什么结论
    \item 具体引用的文献里的一些算法,一些模型,一些基础概念自己是否完全了解为什么是这样的。他们在文章中有什么关联性(与你的核心实验目标)
    \item 有没有之前没发现的表述错误或者图表错误(之后也需要改的)
    \item 你做这项研究的目的是什么,选择某种方法的动机是什么,比其他方法有什么优势
\end{enumerate}

\textbf{二. 整理好上述问题之后,一章章(abstract也含在内)的进行拆分来进行细节分析,包括:}
\begin{enumerate}
    \item 所有背景知识相关介绍的内容,自己能否用自己的话大概表述
    \item 每个小标题表达的内容是什么,整章节内的小标题统一性
    \item 具体实验的或者实验设计的亮点或者核心逻辑
    \item 具体实验结果之间的递进、并列或者其他关系和区别(比如A实验说明了a,B实验说明了b,a是进行实验B的逻辑基础,这就属于递进。如果a和b方向内容相同那就是A和B并列)
    \item 实验对象的选择和实验模型的选择的原因(也包括为啥不用别的相关或者类似的对象和模型)
    \item 基于获得的结果,还有什么是可以未来去尝试的(这个部分可以很多相关问题,包括为什么不做某些实验,未来做某些实验是为什么,未来做某些实验可能能得到什么,未来的某些实验的基础设想和实践难度(也就是为啥你现阶段没做,但是你考虑了))
    \item Abstract要包含关键的结果和创新点
\end{enumerate}

\textbf{三. 最后做ppt和viva的时候的相关重点}
\begin{enumerate}
    \item 一共二十分钟,可以分成四个部分(背景介绍,实验设定,实验结果,推断结论)。每个部分5分钟左右,尤其结尾部分,前面三个部分可以适当简短,毕竟考官都读过你的文章了。
    \item 打印好你的文章带去现场,边问答边做笔记(同时也给自己一些喘息时间(通过记笔记))
    \item 重点图表结果在ppt内展示,并附上简短的结论描述
    \item 不太需要讲遇到的问题,也不需要着重讲未来能做的实验或者设计(考官可能会问到,或者你回答某些问题的时候会答)根据具体情况,这部分可以放弃ppt部分讲,节约时间。
    \item 确保ppt能正常运行,字和图片都看得清,提前测试的时候,尽可能自己模拟实际讲的情况过一遍
    \item 答辩过程中记得喝水,但是慢慢喝,少量多次比较好。开始之前吃一些能量多好消化的食物,实在吃不下也吃点巧克力什么的或者弄点提神的饮品。开始之前多提前点到,熟悉环境有助于放平心态(也记得提前上厕所)
    \item 回答问题的时候可以横跨整个文章,有的章节内的点可能是在后续章节里有联动的,或者未来计划和最终结论相关的。如果遇到考官提出,自己确实没想到或者没做的,该说不知道就不知道,并且提出需求,希望考官在反馈部分明确相关信息的处理办法或者信息来源(方便你后续改的时候知道去哪找东西)。
    \item 一般来说每个问题的问答大概3-5分钟最多,两个考官的情况下,每个老师20个问题就算不少了。所以一般情况下是两个小时到两个半小时,中间会有休息时间的(不过等真的开始了,你应该不太会顾得上时间变化)。
\end{enumerate}
 
\textbf{四. 非论文相关问题}
\begin{enumerate}
    \item 人工智能和Chatgpt能如何inspire你的研究?
    \item 你的研究可以应用到其他类型的数据/疾病/领域吗?为什么?
    \item 你认为你的专业存在什么样的缺点?
    \item 你在读博几年中的最大收获是什么?
    \item 如果再来一次,你还会选择读博并做这个项目吗?
\end{enumerate}
Tips:感觉老师就是比较好奇聊聊,放松回答就行。\textbf{最后来源于PGsupport某老师的数据,咱们建校以来从未有过Fail的答辩结果,因此大家不要太担心,正常发挥就行,祝好运。} 

\textbf{五. 问问自己的老师能不能帮你设想一些问题,另外如果时间充裕,能不能让自己的老师给安排个模拟viva(可选项)。也鼓励去找自己的前辈们在他们答辩结束后去及时咨询相关情况,毕竟每年情况可能多少都会有一些变化。}

\begin{flushright}
    (2024年07月05日 by Ziwen Xie)
\end{flushright}
