%# -*- coding: utf-8-unix -*-
%%==================================================

\chapter{Other Guides}

\begin{flushright}
    ——Take a glance, no loss; take a second look, huge gain
\end{flushright}

\section{A Big Pitfall: What If You Suddenly Lose All Your Computer Data Tomorrow? — How to Back Up Your Files}

\begin{figure}[H]
    \begin{tabular}{rl}
        \includegraphics[width=0.5\columnwidth]{author-folder/Kai.Wu/backup1.jpg} &
        \includegraphics[width=0.4\columnwidth]{author-folder/Kai.Wu/backup2.jpg}
    \end{tabular}
    \caption{\href{https://baijiahao.baidu.com/s?id=1719578217211021768}{Link: A Female University Student in Guiyang Lost Her 8,000-Word Graduation Thesis and Broke Down Crying}}
\end{figure}

Most students buy their own computers during their undergraduate years. Although all their assignments, research data up to the PhD stage, and papers are stored only on this one computer, they are not fully aware of the importance of their computer and the fragility of the information stored on it. Every year, incidents like the one shown above occur, causing heavy blows to those involved, such as making news headlines and being ridiculed nationwide.

Losing a paper you're working on is actually a small matter since all the research data is still there. The person in the news above rewrote the paper in just a few days. But are those extremely important yet seldom-accessed materials absolutely safe?

Imagine if:
\begin{itemize}
    \item One day, your hand slips, the computer falls to the ground, won't boot anymore, and the hard drive can't be recovered.
    \item On a groggy day, you grab a cup of coffee and accidentally spill it on your computer, causing an instant short circuit with sizzling sounds.
    \item You unwittingly click on a link and get infected with ransomware; or the XJTLU intranet is attacked, and all computers are infected with ransomware. All your files are encrypted and locked.
    \item You take your computer out for work or data collection, and the entire computer gets stolen.
    \item On your way home for the holidays, your backpack or suitcase gets stolen or taken by mistake.
    \item Without any malice, but what if XJTLU catches fire, gets flooded (the 5th floor of MB leaks during heavy rain), or the building collapses, and you don't have time to grab your computer during evacuation.
\end{itemize}

\begin{figure}[H]
    \centering
    \includegraphics[width=0.8\columnwidth]{author-folder/Kai.Wu/backup_zhihu.jpg}
    \caption{\url{https://www.zhihu.com/question/394796969/answer/1501840215}}
\end{figure}

Individually, these situations are rare, but adding up all the possibilities over three to four years, the probability of at least one occurring is not small. If it happens, how much will your PhD progress be delayed, and how much of your youth will be wasted?

Having only one copy of your data is extremely fragile. Think now: are the computers, data, and papers you rely on stored only in one place? If so, please get up from your chair immediately, go to the bathroom and wash your face, ask yourself how you can be so careless as an adult, and then start reading the guide below.

\subsection{Theory: The 3-2-1 Principle}

The "3-2-1 Principle" is a common data backup method used by commercial companies, specifically:
\begin{itemize}
    \item 3: Keep 3 complete copies of your files—one original and two backups.
    \item 2: Store the files on at least two different types of media.
    \item 1: Keep one backup copy off-site.
\end{itemize}

The "media" refers to different storage mediums such as internal hard drives, external hard drives, USB drives, cloud drives, etc. It may sound complicated, but the simplest practice for us is: keep the original on the internal hard drive, and store two backups on an external hard drive and a cloud drive respectively, thus meeting all three requirements simultaneously. This way, you can handle all the catastrophic problems listed above (even if the cloud drive company suddenly goes bankrupt) without affecting your research.

\subsection{Practice}

Now it's time to get started. First, decide the scope of your backup based on your resources. In the most extravagant scenario, you can make a local backup of your entire computer plus a cloud drive backup. However, often the cloud storage space is limited, external hard drive capacity may be insufficient, or the network speed is too slow to upload too much data. So, first define the scope of your backup.

\begin{itemize}
    \item The most economical solution: Only back up the most important files, such as papers, research data, CV, and desktop. Use real-time cloud backup + periodic external hard drive backup. This requires very little capacity on both the external hard drive and cloud drive.
    \item The most efficient solution: Locally back up the entire computer (including the most important data); add one more real-time cloud backup for the most important research papers, etc. This requires an external hard drive larger than your computer's hard drive (not expensive), plus a normal capacity (about 10 GB) cloud drive.
    \item \sout{The most extravagant solution: Buy a NAS, insert several tens of terabytes of drives, set up RAID 1, and then synchronously transfer a copy to an off-site location.}
\end{itemize}

A typical computer hard drive capacity is generally within 2 TB. I assume it's not too much to ask for everyone to buy an external hard drive of 2 TB or more (within 500 yuan). If your supervisor has sufficient funding, you can directly ask for one—just say you want an external hard drive for backup and see if they can help you purchase one. Reimbursements within 2,000 yuan are quite easy. Therefore, I'll only introduce the efficient method below.

\subsubsection{Full Computer Backup}
\label{sec.pc_backup}

One major benefit of making a full backup of your computer is that if your computer is completely lost, you can fully restore your previous working state within one day after buying a new computer, seamlessly resuming your work without affecting your research.

Windows and macOS both come with built-in system-level full backup solutions. There are many tutorials online, so I won't elaborate.

\begin{itemize}
    \item For Windows, search "Windows backup" on Bilibili for tutorials. Generally, use the system's "Backup and Restore (Windows 7)" (although it says Windows 7, it works perfectly on Windows 10 and Windows 11, which is why it hasn't been discarded). For example, watch this video: \url{https://www.bilibili.com/video/BV1Dy4y1x7RP}
    \item For macOS, search "Mac Time Machine" on Bilibili. Generally, use "Time Machine" in System Settings. For example, watch this video: \url{https://www.bilibili.com/video/BV1oy4y177qS}
\end{itemize}

If one video isn't clear, don't worry—search for a few more to watch, and also make good use of Google and Zhihu. Beginners may feel that the learning cost is a bit high, but don't wait until your computer crashes to start preparing.

\subsubsection{Cloud Drive Backup}
\label{sec.net_drive_backup}

Box (\url{https://box.xjtlu.edu.cn/}) is the school's self-built cloud drive, upgraded in 2022 to provide 100 GB of free space per person. Upload and download speeds within the intranet can reach gigabit levels, and it comes with three months of file history versions, outperforming many other cloud drives. Drag important folders into the Seafile client (without moving folders), and it will automatically back up to the school's cloud drive in real time. Usage instructions can be found at \url{https://guide.xjtlu.edu.cn/box/student/drive_client/drive_clent_for_windows.html}. The only problem with Box is that it does not support file-sharing links (teachers have sharing permissions; students do not), but it's very suitable as a backup cloud drive.

Additionally, if you don't fully trust the school's IT, you can also install another commercial cloud drive for a second layer of cloud backup. I personally strongly do not recommend Baidu Cloud because it occasionally loses files randomly. Among domestic cloud drives, I recommend Nutstore (free version has unlimited space but limits monthly traffic). For foreign cloud drives, I recommend OneDrive (5 GB free for personal accounts; you can search on Taobao to expand OneDrive storage to 15 GB permanently for a few yuan). Additionally, everyone has a free 1 TB of OneDrive storage under their Liverpool account—take advantage of it (but remember to migrate your data before graduation because the account will be reclaimed). Refer to this section \ref{sec.fuli_liverpool}.

\subsection{Paper Backup and File History Versions}
\begin{minipage}[t]{0.65\textwidth}
    The backup methods mentioned above can handle most data loss scenarios. However, your papers, including dissertations and journal articles—the most important yet fragile materials—not only fear loss but also especially fear these two devastating disasters:
    \begin{enumerate}
        \item Version confusion. It's common to modify a paper dozens of times from the beginning to completion. Especially when nearing submission, you'll inevitably battle through a dozen more versions due to feedback from your supervisor, co-supervisor, and collaborators. The differences between versions may be minimal and hard to distinguish. There will inevitably come a day when you can't figure out which version is which or forget what modifications were made in which version, leading to repeated time spent distinguishing versions, which in turn leads to ↓
        \item File overwriting, such as (1) a new version directly overwriting the old version, but after a while, you need to retrieve the old version for some reason; (2) the old version overwriting the new one, wasting months of revisions.
    \end{enumerate}
\end{minipage}
\begin{minipage}[t]{0.34\textwidth}
    \begin{figure}[H]
        \includegraphics[width=0.95\columnwidth, right]{author-folder/Kai.Wu/thesis_versions.jpg}
    \end{figure}
\end{minipage}

At this point, if you're interested, you can do an experiment: create a file and write some content in it. Then, in another folder, create a file with the same name and write different content. Finally, drag it over to overwrite the previous file. Now, see if you have any way to retrieve the content of the previous file.

You might ask, didn't we mention so many backup methods above? Yes, but those backup methods mainly serve to prevent loss, like the computer being lost or hard drive failure, but few backups can prevent overwriting. For example, real-time backup cloud drives: if you mistakenly overwrite the file locally, it will be immediately overwritten in the cloud as well. Local backup hard drives: if the backup frequency isn't high, there's still a chance, but if you look for it after a while, you might not find it.

Therefore, for the safety of your paper (your lifeline), it's entirely worth adding another layer of insurance:

\begin{enumerate}
    \item Enable system-level File History protection in Windows: \url{https://www.asus.com.cn/support/FAQ/1013067/}. Don't forget to add your paper folder to make it effective.
    \item macOS's Time Machine backups come with a file history function. Refer to the previous section \ref{sec.pc_backup} to configure Time Machine, and the file history of your entire computer will be included in the backup.
    \item Linux users, I'm sure you have your methods. The simplest way is to write a shell script that directly copies the paper folder to another directory, naming it with the date. Finally, add this script to crontab to run periodically.
    \item The above are methods to enable local file history. Some cloud drives come with file versioning functions but usually limit the number or dates of versions or require a membership purchase to enable file versions. After enabling them, it's equivalent to also storing a version history in the cloud. In case your local settings aren't properly configured or can't be used for some reason, it can save you when you mistakenly overwrite files. The school's Seadrive comes with about two months of file versions, free and very generous. Again, I recommend everyone use it. Just drag the folder in to back up, and it will automatically generate file versions. See section \ref{sec.net_drive_backup}.
    \item For those who know a bit of Git and GitHub, please create a Git repository for your entire paper folder and upload it to GitHub as a private repo. Commit once every day after work to back up versions and record what content you modified. For papers, Git is the most perfect backup solution, allowing you to easily roll back to older versions without ever worrying about overwriting, and even easily know when a change was introduced, more reliable than all the above methods. Overleaf also has GitHub integration, making it convenient to collaborate with supervisors. But Git has a slightly higher learning cost, but there are countless detailed tutorials online. Interested and capable students can learn on their own.
\end{enumerate}

\begin{flushright}
    (December 2, 2022 by \Wu)

    (Major update: December 30,

    (major update: 2022年12月30日 by \Wu)

    Translated by GPT
\end{flushright}

% \begin{figure}[H]
%     \centering
%     \includegraphics[width=0.5\columnwidth]{author-folder/Kai.Wu/}
% \end{figure}


% \usepackage[export]{adjustbox}

% \item 
% \begin{minipage}{0.3\textwidth}
%     文字
% \end{minipage}
% \begin{minipage}{0.63\textwidth}
%     \begin{figure}[H]
%         \includegraphics[width=0.95\columnwidth, right]{author-folder/Kai.Wu/}
%     \end{figure}
% \end{minipage}

% \input{author-folder/Kai.Wu/.tex}
 \clearpage

\section{Remote Control: How to Access Campus Computers from Off-Campus}

When you are off-campus, you might often miss your campus computer. Whether it's the desktop provided by the school or other computers bought with your advisor's funds, they can all be remotely controlled.

\subsection{Preparation for School-Provided Computers}
(If you are not trying to access a school-provided computer, please skip to the next section)

School-provided computers are quite special, and by default, we cannot install software on them. There are two ways to handle this:
\begin{enumerate}
    \item Email or call IT and directly ask for administrator privileges. Just say you are a PhD student and need to install software on the school-provided computer. After that, you can install the control software.
    \item After the above method, the computer is still managed by the school and there are still some restrictions, but it is generally sufficient for regular use. For those who want complete control over the school computer, you can format the disk and reinstall the system, or partition the disk (keeping the original system) and install a dual system. There are many tutorials on reinstalling Windows, partitioning, and setting up dual systems on Bilibili and Zhihu.
\end{enumerate}

\subsection{Recommended Desktop Control Software}
The following software can be used on any system: Windows, Linux, or Mac.
\begin{itemize}
    \item VNC: Install VNC Server on the controlled end (school computer) \url{https://www.realvnc.com/en/connect/download/vnc/}, and VNC Viewer on the controlling end (your computer) \url{https://www.realvnc.com/en/connect/download/viewer/}. Register an account and log in to use, no need to remember connection codes. Recommended because VNC is a well-tested remote desktop solution that generally works as long as the network is stable.
    \item ToDesk: \url{https://www.todesk.com/} Install the same software on both computers. Recommended because it is a new software and generally smoother than VNC.
    \item Others: AnyDesk, TeamViewer, Sunflower, all can be tried. However, TeamViewer is not highly recommended because it might detect the network environment of the school and force you to purchase a commercial version, which is not an issue with other software.
\end{itemize}

\subsection{Remote SSH}
(If you don't use Linux, you can skip to the next section on pitfalls)

If the controlled end is a Linux system and you mainly use command-line operations, connecting via SSH directly will be much faster than desktop control.

The computer on campus has an internal IP starting with 10. How to SSH from outside? This trick is called "NAT traversal."

First, I recommend this video:

\href{https://www.bilibili.com/video/BV1Qq4y1w7F5}{[Hardcore] Public Network Access? NAT Traversal! Zero Experience Start!}\url{https://www.bilibili.com/video/BV1Qq4y1w7F5}

The solutions available for campus machines in the video are:
\begin{itemize}
    \item At 04:52 in the video, using IPV6 connection. However, note that in my experience, the school's IPV6 is unstable and might stop working after a few days. Even if you can set up DDNS, I personally do not recommend it.
    \item A more reliable solution is introduced at 08:09 in the video: NAT traversal. Free and easy-to-use solutions include: Zerotier (slightly slow as it is overseas), Peanut Shell (an old domestic brand, but registration requires an ID card), NOFRP (new, reliability might be low), or directly search "free NAT traversal" on Bilibili for many new solutions. If you are not familiar with these, there are many tutorials on Bilibili and Zhihu.
    \item (Advanced high-traffic version, but with a higher learning curve) Free solutions have limitations on traffic and bandwidth, which are sufficient for SSH commands. But if you want to use SCP to transfer files or forward VNC via SSH, consider renting a cloud server with your advisor's funds and manually setting up an FRP service (a bit complicated, but very smooth once set up, faster than the previous remote desktop solutions). Alternatively, set up a junk machine as a jump server in your dorm with a public IP and DDNS for unlimited traffic access. These solutions have a higher learning curve, but you can refer to online tutorials and experiment slowly.
\end{itemize}

\subsection{Pitfalls}
Here are some experiences I gained after encountering many pitfalls:
\begin{enumerate}
    \item Redundancy to improve reliability: Do not rely on a single remote control solution. If one fails, you can use another. Please install at least two, and if you are away for a long time, install more. I personally use: a second-hand thin client at home with FRP and DDNS + Peanut Shell as a backup plan + VNC as another backup plan + ToDesk, a four-layer backup solution for stable access to my campus Linux computer.
    \item After installation, restart once to see if the control software can start automatically.
    \item If the controlled end is a Windows computer, be sure to disable system updates. Although restarting is fine, a major Windows update might get stuck on a screen asking you to agree to a new user agreement, and unless you are there to click it, no software will start, and you will lose control. Please search "disable Windows updates" on Baidu. I personally recommend using a combination of group policy and host file modification.
    \item For those running programs, be sure not to exceed memory limits or crash the system memory. If the memory is fully used, the system will crash immediately, killing all control software, and it will not restart automatically. You must be there to force a restart. Please find ways to monitor and control memory usage in your programs.
    \item If possible, purchase remote KVM hardware like "Sunflower Control" to force a remote restart even if the system crashes.
    \item No matter how good the solution is, it cannot prevent network maintenance or power outages on campus. These can be defended against. For network outages, make sure to check the macauth option when logging in to the network, so it will work again after maintenance. For power outages, the motherboard needs to support the "boot on power" or "boot with power" function. Check the motherboard manual or search for "power" to see if it is available. Alternatively, purchase Sunflower Control. However, power and network outages are rare, and unless you are away for months, you generally do not need to worry about them.
    \item Maintain a good relationship with at least one on-campus classmate. Newcomers will encounter various unexpected situations and lose control, requiring office classmates to help check the machine. Behind 100 remote control solutions, you also need good classmates for physical control. In case the network cable is accidentally kicked out, you still need a classmate to help you fix it.
\end{enumerate}

\begin{flushright}
(09 November 2022 by \Wu) \\
Translated by GPT
\end{flushright}
 \clearpage

\section{My Lower Back Can't Take It Anymore - How to Maintain Cervical and Lumbar Health}

\subsection{Relieving Back Problems: Standing Desk}

\begin{minipage}[t]{0.39\textwidth}
    \begin{figure}[H]
        \includegraphics[width=0.95\columnwidth, right]{author-folder/Yue.Zhou/gongzuotai.jpg}
    \end{figure}
\end{minipage}
\begin{minipage}[t]{0.6\textwidth}
    Before pursuing my PhD, I already had severe lumbar muscle strain. This condition, while not a major illness, constantly torments you. At its worst, the pain kept me awake at night. I visited the hospital and got medication, but the doctor said this condition requires self-care and cannot be completely cured. Later, I tried running for half a year, and miraculously, the lumbar muscle strain improved, and now it doesn't hurt at all (though prolonged sitting still causes pain). After starting my PhD, I lost the motivation to exercise, so I bought a standing desk (highly recommended for everyone! It's so useful!). You can work standing up, and when you get tired, you can sit down. This greatly reduces the damage to the lumbar spine caused by prolonged sitting.
\end{minipage}

\begin{flushright}
(December 30, 2022 by Yue Zhou)

Translated by GPT
\end{flushright}

\subsection{In-depth Advice from a Severe Lumbar Disc Herniation and Cervical Spondylosis Patient}

\subsubsection{Standing Desk: Can Greatly Relieve Lumbar Problems, But Not Completely Solve Them}

I've had lumbar disc herniation for ten years and cervical spondylosis for five years. In 2019, my back pain was so severe that I couldn't sit for a whole day, so I put a standing desk in the lab and used it for three years. This year, I also put the same standing desk in my bedroom. However, standing for long periods is just as bad for your back as sitting, and eventually, standing and sitting will both cause back pain. A standing desk is not a long-term solution; standing just changes the angle of stress on the lumbar spine.

You can't rely entirely on a standing desk: Over the years, I've relied too much on the standing desk, and now standing for long periods causes back pain just like sitting. When I went for an MRI, the doctor even asked if my back was fractured. To really relieve back pain, you need to avoid prolonged sitting and standing. High school teachers often suffer from back pain partly because they stand for long periods while teaching.

In summary, a standing desk can be used, but it shouldn't replace sitting at a computer for long periods. Alternate between standing and sitting, and exercise regularly.

Note: Remember to adjust the height of the standing desk occasionally. The standing desk I put in the lab hasn't been adjusted for years, and it seems the hydraulic rod has rusted and won't move, turning it into a regular desk.

\subsubsection{External Monitor: Avoid and Relieve Cervical Spondylosis}

The uncomfortable thing about cervical spondylosis is that prolonged looking down causes neck pain. When I'm out on the subway or high-speed train, I have to hold my phone up to eye level, or I can't use it for long.

Another way to avoid and relieve cervical spondylosis is to avoid using a laptop for long periods. Prolonged looking down at the screen can lead to cervical spondylosis. The same goes for looking down at your phone.

If you use a laptop frequently, place a desktop monitor in both your bedroom and office as an external monitor.

I developed cervical spondylosis after a year of intensive laptop use, and it stopped worsening after I switched to using an external monitor.

If you frequently use a laptop intensively without an external monitor, it's just as bad for your neck as looking down at your phone all day. I often advise people in the lab who use laptops for long periods to use an external monitor.

A laptop stand can also help and is cheaper than a monitor. But if you have poor eyesight, it's better to use an external monitor because laptop screens are too small and strain your eyes. I used to buy many laptop stands, but now I use them to hold monitors, and the laptop fits nicely underneath.

\begin{figure}[H]
    \caption{In my bedroom, I use a standing desk with a laptop stand. This combination works well for someone like me who mainly uses a laptop.}
    \includegraphics[width=0.95\columnwidth, center]{author-folder/Jialin.Wang/shengjiangzhuo.jpeg}
\end{figure}

% \subsubsection{Further Relief: Using the Computer While Lying Down}

% Since the beginning of this year, I can't stand for long periods, so I've started exploring using the computer while lying down or prone. I tried lying down for a few days but gave up because it made me sleepy. I think lying down is more suitable for watching movies or playing games.

% I've been trying the prone position for a month now, and it feels okay. At least propping up my upper body doesn't make me sleepy.

% I placed a portable monitor by the bed and used a prone pillow, which allows me to lie down all day. The prone pillow is great because it doesn't compress the abdomen after meals, causing discomfort.

% \begin{figure}[H]
%     % \centering
%     \includegraphics[width=0.3\columnwidth, center]{author-folder/Jialin.Wang/pazhen.jpeg}
% \end{figure}

% Then I use an adjustable laptop stand for the monitor, keyboard, and mouse—though you can just use a laptop, but it's inconvenient due to the heat and moving it around. A portable monitor allows quick switching between the standing desk and prone position—I also considered buying a regular monitor stand, but there aren't any suitable for bed use, and regular monitors are too big for bed use. A portable monitor, which is a repurposed laptop screen, is just the right size for bed use.

% If you use the computer while lying down for long periods, I found it's best to use your arms to slightly support your upper body. But the main support should come from the torso, with the arms providing auxiliary support to avoid too much pressure on the chin.

% I've now mastered the ability to lie prone all day. As long as the laptop stand is adjusted to the right angle, keeping the hands and arms parallel, the wrists won't get sore from prolonged bending.

% However, since I've gotten used to lying down and it's cold, I don't exercise at all. If I don't lie down, my back still hurts. Regular exercise is essential.

% This setup is perfect for someone like me who can't sit or stand for long periods due to back pain, but it won't cure the pain. Exercise is the most important thing. Without regular exercise, unused muscles will atrophy, making things worse. (I wish I could live in a tropical coastal area and swim at the beach every evening; the sea water is warm in the evening.)

\subsubsection{The Best Solution: Change Lifestyle Habits and Exercise More}

For those with lumbar disc herniation and cervical spondylosis, regular exercise is recommended. But don't blindly go to the gym; the best exercise is swimming because most other exercises require using the back muscles, which can worsen the pain.

Currently, I only occasionally have back pain that prevents me from bending over. I know someone whose back pain was so severe that they walked with a limp, but they gradually improved through exercise.

If you lack self-discipline, joining a school swimming club can help, as having people to exercise with regularly can be motivating.

If you get a membership, I remember the Dushu Lake Swimming Pool offers half-price annual memberships every September, but school clubs are cheaper and more cost-effective. Before I got a swimming membership, I used to go swimming once or twice a week with the school swimming club. After getting the annual membership, I went less than five times a year, and even less frequently afterward.
If you have self-discipline, daily exercise is the best method.
If you can't stick to it and can't change your lifestyle habits, the condition will worsen.

Finally, if the back pain is severe, visit the hospital's orthopedics department for an MRI to see the extent of the lumbar spine condition and determine the specific recovery plan. Back pain is now common among young people.

\begin{flushright}
(December 30, 2022 by Jialin Wang)

Translated by GPT
\end{flushright}
 \clearpage

\section{Liverpool Visit}
\label{section.UoL_visit}

\subsection{Potential Opportunities}
\subsubsection{Collaboration with Liverpool Supervisors}
Although most students will have a supervisor from Liverpool on their team, each project may be different, and you may not have frequent interactions with them before going to Liverpool. This infrequency could be due to various reasons, not necessarily because your work is in a different field. So if you think it would be interesting to collaborate with them in Liverpool, please prepare in advance.

If you haven't had much interaction with them before, make sure to establish some familiarity through some means, which will make future collaboration smoother. This can be through video meetings or emails sharing your research progress and plans, co-authoring papers, or brainstorming potential collaborative research content in advance.

We know that teamwork can happen for different reasons (see \ref{subsection.teamwork}), but in any case, be sure to confirm at least the general direction of your activities during your visit with your Liverpool supervisor in advance, including but not limited to: things that require their involvement, schedule, things you need to prepare, and things you plan to do.

\subsection{Accommodation and Living}
\href{https://hallslife.liverpool.ac.uk}{Halls Life} is the community hub for student accommodation at the University of Liverpool, providing various suggestions on accommodation and living. It is strongly recommended to browse it before departing for the UK, as it can help decide whether to bring certain luggage.

\subsubsection{Visiting Other University Libraries}
When you are traveling and visiting various universities in the UK, if you happen to pass by a university library when you are tired, you can actually go in and sit down, as long as you have applied for \href{https://access.sconul.ac.uk/sconul-access}{SCONUL Access} in advance.

Although different libraries have different policies, in most cases, you can at least enter the library after successfully applying, and some libraries may also provide computer access or book borrowing privileges.

\begin{quote}
    Q. Do I need to complete an online application for each institution I would like to visit?
    
    A. No, although you do need to indicate a library you are interested in visiting as part of your application, your approval email allows you to visit any of those libraries on the list. You do not need to reapply. \hfill 21 Oct, 2024 from \href{https://access.sconul.ac.uk/page/access-faq#Multiple%20applications}{Access FAQ | SCONUL}
\end{quote}

After submitting the application, it will be processed by the University of Liverpool Library, and once approved, you can go to the front desk of each university library with the email to get a visitor card.

\subsection{Food and Entertainment}
\href{https://hallslife.liverpool.ac.uk}{Halls Life} is the community hub for student accommodation at the University of Liverpool, providing various suggestions on food and entertainment.

\begin{flushright}
    October 21, 2022 by \Shiyao

    Translated by GPT
\end{flushright}
 \clearpage

\section{XJTLU Virtual Machine}
\url{https://vdi.xjtlu.edu.cn/portal/webclient/index.html}
If you are outside the university and want to quickly access on-campus resources and software temporarily, you can use the university's online virtual machine. The virtual machine is equivalent to a computer located on campus.

\section{Library Literature Databases}
After connecting to the campus network, most of the journals subscribed by the university can be directly downloaded from the journal pages. However, (1) some journal websites cannot recognize your IP, such as the famous Annual Review series; (2) if you want to download papers off-campus, you can use the library's literature databases.

\begin{figure}[H]
    \includegraphics[width=0.7\columnwidth, center]{author-folder/Kai.Wu/library-ejournal.jpg}
\end{figure}

What I use most is clicking e-journal on the homepage \url{https://lib.xjtlu.edu.cn}, searching for the journal by name, and then accessing it through a dedicated link. The library also provides many full-text databases, including CNKI and other Chinese journal databases. For detailed usage, see the library's past training sessions \url{https://core.xjtlu.edu.cn/course/view.php?id=905}, or directly find the PPT in this project's GitHub (\href{https://github.com/xp-pgrs-unofficial-guide/xp_pgrs_unofficial_guide/tree/main/fileshare}{link}).

\section{Borrowing Books from Suzhou Library and Free Book Delivery}
If you want to read books from Dushu Lake Library or borrow books from Suzhou Library but don't want to go there because it's too far, what should you do? Just use the mini program to borrow books! Books from Suzhou Library can be \textbf{[free of charge]} delivered to Dushu Lake Library for pickup, and books from Dushu Lake Library can be directly delivered to the entrance of the university library.

Search for the mini program "Suzhou Library Borrow Books" on WeChat. Since the medical insurance card is the citizen card, you can use it to borrow books directly. If you have already obtained the medical insurance card, you can bind it directly to waive the card opening fee. Then follow the prompts in the mini program.

% \section{Student Medical Insurance}
\section{Pay Attention to Your Negative Emotions and Depression Tendencies}
\label{sec.mental_health}

\begin{figure}[H]
    \includegraphics[width=0.6\columnwidth, center]{author-folder/Kai.Wu/mental_1.pdf}
    \includegraphics[width=0.6\columnwidth, center]{author-folder/Kai.Wu/mental_2.pdf}
    \includegraphics[width=0.6\columnwidth, center]{author-folder/Kai.Wu/mental_3.pdf}
\end{figure}

The \href{http://ga.berkeley.edu/wellbeingreport}{Graduate Student Happiness and Well-Being Report [link]} published by the Berkeley Graduate Assembly in 2014 shows that 46\% of Ph.D. students have mental health problems. Studies published in authoritative journals such as Nature suggest:
\begin{enumerate}
    \item Graduate students are six times more likely to suffer from severe anxiety and depression than the general population \href{https://www.nature.com/articles/nbt.4089}{[reference link]}
    \item Smarter people are more likely to suffer from mood disorders such as anxiety or depression but are less likely to seek help \href{https://www.nature.com/articles/nj7677-549a}{[reference link]}
    \item Mood disorders account for 80-90\% of suicides \href{https://www.sciencedirect.com/science/article/pii/S0160289616303324}{[reference link]}
    \item During the COVID-19 pandemic, four out of five researchers showed signs of mental health distress \href{https://www.tandfonline.com/doi/full/10.1080/21568235.2021.1992293}{[reference link]}
\end{enumerate}

\begin{figure}[H]
    \begin{tabular}{rcl}
        \includegraphics[width=0.31\columnwidth]{author-folder/Kai.Wu/chat1of3.jpg} &
        \includegraphics[width=0.31\columnwidth]{author-folder/Kai.Wu/chat2of3.jpg} &
        \includegraphics[width=0.31\columnwidth]{author-folder/Kai.Wu/chat3of3.jpg}
    \end{tabular}
    \caption{Chat records with a graduated classmate. The incidence of depression among our university's Ph.D. students is higher than imagined}
\end{figure}

As a high-risk group, we have received extensive professional knowledge and skills training, yet almost no one has taught us "how to maintain mental health." If you often feel nervous, anxious, or disappointed now, and even have symptoms like insomnia, please immediately face your situation, put your health first, and feel free to seek help from the university's psychological counseling center.

Meanwhile, prevention is always more effective than post-treatment. I hope that everyone reading this, whether you feel good now or not, will click the links below, read Zoe Ayres' lecture in Liverpool (the source of the above PPTs), and her book "\textit{Managing Your Mental Health During Your PhD - A Survival Guide}," to get through your graduate studies healthily. ~~\href{https://github.com/xp-pgrs-unofficial-guide/xp_pgrs_unofficial_guide/tree/main/fileshare}{[GitHub link]}~~~~~\href{https://gitee.com/kaiwu-astro/xp_pgrs_unofficial_guide/tree/main/fileshare}{[Use this link if you can't access GitHub]}

\section{What Software Have You Installed on Your Computer? (Looking Forward to Your Additions)}

\KW: Let me start by sharing some software that I would definitely recommend to my fellow junior students:
\begin{figure}[H]
    \includegraphics[width=\columnwidth]{author-folder/Kai.Wu/kai_mac_dock.jpg}
\end{figure}
\begin{itemize}
    \item Research and Productivity Tools
    \begin{enumerate}
        \item Notion (All platforms). An incredibly powerful new-generation note-taking app. I was originally a heavy OneNote user, but after being introduced to Notion by chance, I decided to abandon OneNote and completely switch to Notion after a short trial. Recommended because: supports Win/Mac/Linux/Web/iOS/Android; powerful database functions to perfectly build your own GTD system and greatly improve efficiency; fast global search; modern collaboration features, even usable for team and project management; powerful sharing features to easily share notes as public webpages; lots of tutorials on Bilibili.

        \includegraphics[width=0.8\columnwidth]{author-folder/Kai.Wu/notion.jpg}
        \item TickTick (All platforms). I've used Wunderlist (later acquired by Microsoft and became Microsoft To-Do), iOS's built-in Reminders, and several other list apps for a long time. Only TickTick feels the most intuitive and user-friendly. Used together with Notion, it can conveniently manage all aspects of study and life. Supports all platforms.

        \includegraphics[width=0.8\columnwidth]{author-folder/Kai.Wu/dida.jpg}
        \item Trello (All platforms). Manage your to-do items and tasks with cards. My supervisor heavily relies on Trello to keep his numerous tasks well-organized. While I think Notion is a superset of Trello, if you don't like Notion for some reason, at least try Trello. Available for download on all platforms.

        \includegraphics[width=0.8\columnwidth]{author-folder/Kai.Wu/trello.jpg}
        \item VSCode (Visual Studio Code) (All PC platforms). An open-source text editor that rose to prominence around 2019, with extremely strong community and plugin support. Currently, I use VSCode + LaTeX Workshop plugin for my journal papers, theses, and even this guide you're reading. All my numerical simulation programs and data analysis scripts are also done with VSCode. The drawbacks are a small learning curve and the need to configure settings; it also consumes memory like Chrome.

        \includegraphics[width=0.8\columnwidth]{author-folder/Kai.Wu/vscode.jpg}
        \item Spark. An email app developed by Readdle, an award-winning Apple app developer, that manages all your email accounts in one place and syncs email settings to the cloud, solving the problem of re-adding multiple email accounts when you get a new device. It seems to be exclusive to Mac and iOS; not sure if it's available on other platforms.

        \includegraphics[width=0.8\columnwidth]{author-folder/Kai.Wu/spark.jpg}
        \item As for reference management software, I won't make recommendations—everyone has their own preferences. If you don't have one yet, ask your seniors or check out recommendations on Bilibili.
    \end{enumerate}
    \item Efficiency Tools
    \begin{enumerate}
        \item Bob (MacOS) or Pot (Win/Mac/Linux). Tools for word translation, screenshot translation, and OCR. Supports integration with ChatGPT, DeepL, Google Translate, Youdao Translate, and other translation services, and can even display results from multiple platforms simultaneously for comparison. Bob can be downloaded from the Mac App Store; Pot is available at pot-app.com.

        \includegraphics[width=0.8\columnwidth]{author-folder/Kai.Wu/Bob.jpg}
    \end{enumerate}
\end{itemize}

Have you discovered any great software? Please feel free to share them with your fellow junior students.

Your recommendations:

\begin{flushright}
    Translated by GPT
\end{flushright}