
\chapter{前言}
\thispagestyle{empty}
{\kaishu{我之前跟一些博士生同学交流的时候,他们有的都快毕业了,有的都不知道可以去领办公用品,还有的甚至学校体育馆在哪里都不清楚,更别说免费的健身房、球场资源。以及在博士生微信群里,每年都有人到了第一次APR的时间才知道还有meeting record需要填。再以及,利物浦账号下的大量数字资源,了解的人就更少了。有些重要的事,自己不知道,导师不知道,办公室同学也不知道,那最后真的就可能等到毕业了才知道。

这就是发起本攻略项目的目的。虽然说每个同学入学的时候都收到了一本学校发的PGRS Handbook,但很遗憾的是,远远不够。一来其中缺少很多内容,例如利物浦meeting record、利物浦账户福利几乎没介绍;二来由于是全英文,翻了几页干货又不多,估计很多同学像我一样直接就失去了阅读兴趣;三来入学时发了一大堆材料,虽然官方手册有其重要性,但很容易就淹没在那一堆材料里了。

对我个人而言,入学的时候也是在各位办公室学长的帮助下才获得了很多必要知识。因此我就想到可以克服原手册的弊端,把这些经验记录下来,以泽后人。与我而言,也是为西浦集体做出的微小贡献。也希望受益于本手册同学,能在学习中把自己宝贵的经验记录下来,将帮助传递下去。

本手册并不想替代PGRS Handbook,而只是作为补充以及重点摘录。因此希望看到这本书的同学,有空的时候还是把官方手册从垃圾桶里捡起来,很可能有重要的发现。

如果想表达你对作者们的感谢,可以到本项目GitHub上点亮一个star(链接:\url{https://github.com/xp-pgrs-unofficial-guide/xp_pgrs_unofficial_guide})。同时,如果你受到了此文档的帮助,也鼓励你花几分钟把你的亲身经验记录下来,把帮助传递下去。如何投稿,请见本文档第 \ref{chapter.author-ins} 章 \space 写作指南。

最后,本手册完全用爱发电,从未受到任何资助,也没有官方背景。其中可能有谬误和缺漏,敬请原谅,欢迎指正。

\begin{flushright}
项目发起人~~~\KW\\
2022年8月18日于MB
\end{flushright}

}}
