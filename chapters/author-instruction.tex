\chapter{Writing Guide}
\label{chapter.author-ins}

% v0.1 by Kai.Wu

The experience you write down in a few minutes may save future readers hours or even days. You are welcome to contribute to this guide.

The simplest and most straightforward way to contribute is to send plain text, Word, or TeX files directly to \Shiyao. The administrator will help you convert the files into \LaTeX{} format and include them in the guide. Although the administrator might not be very happy about it because it takes time to create, organize, and upload the files, it is still acceptable. If you feel sorry for the administrator, you can look at the following methods (those who find it troublesome can skip the following part and go directly to the last paragraph of this page). A better way to contribute:

\begin{itemize}
    \item If you have experience using Git and submitting Pull Requests, and you can write \LaTeX{}, you can directly submit a PR on the \href{https://github.com/xp-pgrs-unofficial-guide/xp_pgrs_unofficial_guide}{GitHub page of this manual}. \href{https://www.zhihu.com/question/21682976/answer/79489643}{Refer to this link for how to submit a PR}
    % \item If you don't know how to submit a PR but can write \LaTeX{}, you can write on Overleaf. Please send an email to Kai.Wu19 at student.xjtlu.edu.cn to request editing permissions. After writing, please email the administrator to push the manuscript to GitHub.
\end{itemize} 

\vspace{5mm}
Note: For the advanced contribution methods above, to avoid messing up the order due to multiple collaborations, each author should create their own subdirectory under the \texttt{author-folder} folder, create a new \texttt{tex} file in it, and start writing directly from the \texttt{section} command on the first line, followed by the main text. After finishing, insert your manuscript into the correct chapter in the \texttt{chapters} folder using the following code:
\begin{lstlisting}
    \input{path to your tex file}
\end{lstlisting} 
For example, if Duoyu Wang wants to share his commonly used software, he should first create a \texttt{duoyu.wang} folder under \texttt{author-folder}, create a new \texttt{health-insurance-use.tex} file in it, write \lstinline[breaklines=true]!\section{Using Student Health Insurance}! as the title on the first line, and then write the main text. After finishing, add a line \lstinline[breaklines=true]!\input{author-folder/duoyu.wang/health-insurance-use.tex}! in the \texttt{others.tex} file in the \texttt{chapters} folder, and it's done.

\vspace{5mm}
For some advanced editing methods, refer to the template manual and examples in the \texttt{author-guide} directory of the project GitHub.

\vspace{5mm}
You can choose to be anonymous or include your name, email, department, and year of enrollment. If you wish, you can even include your birth date, birth time, property details, etc., and maybe ask the PGR Society if they can arrange a blind date (just kidding). Finally, a reminder: You must not publish any content that violates the laws of the People's Republic of China, the PhD Student Code of Conduct of Xi'an Jiaotong-Liverpool University, the \href{https://www.liverpool.ac.uk/aqsd/academic-codes-of-practice/pgr-code-of-practice/}{PhD Student Code of Conduct of the University of Liverpool}, or any other applicable regulations. You are responsible for the content you publish, and the PGR Society cannot guarantee or review the accuracy of the content. The PGR Society and other authors are not liable for any disputes arising from the content you publish.

\vspace{5mm}
Thank you again for your contribution to the juniors and the future of our country (ง •̀\_•́)ง

\begin{flushright}
    Translated by GPT
\end{flushright}